\chapter{Исследовательский раздел}

\section{Исследование эффективности разработанного метода}

Метриками эффективности метода были выбраны размер избыточной информации и возможность восстановления файла. 
Избыточной информацией является суперблок.

\begin{flushleft}
В рамках исследования были выбраны несколько видов файлов:
\end{flushleft}
\begin{itemize}
\item текстовые файлы;
\item изображения;
\item видеозаписи.
\end{itemize}

Для проверки следует сравнить, насколько размер суперблока больше или меньше размера исходного файла.

Для этого сравнения были сделаны резервные копии файлов, типы которых выделены выше. 

\begin{flushleft}
По результатам анализа были сделаны следующие выводы:
\end{flushleft}
\begin{itemize}
\item суммарный размер суперблоков равен размеру исходного файла;
\item метод гарантирует восстановление файла при потере одного сервера, если количество серверов больше одного.
\end{itemize}

Рассмотрим пример с файлом, который разделен на три блока. Суперблоки распределены по трем серверам.

При отказе одного из серверов, файл можно восстановить при помощи суперблоков, сохраненных на другие сервера.

Чтобы добиться такого уровня отказоустойчивости в механизме полной копии нужно сохранить на все сервера копию файла, в результате чего количество избыточной информации будет в 3 раза больше, чем с использованием разработанного метода.  

%В результате анализа была выведена форма количества суперблоков, которые создаются при резервировании:
%\begin{equation}
%SuperblockN = \left\{\begin{matrix}
%1, BlockN = 1, \\
%BlockN - 1, BlockN \in [2,\inf]
%\end{matrix}\right.
%\end{equation}
%где $SuperblockN$ -- количество суперблоков, $BlockN$ -- количество блоков, на которые был разделен файл.  


%Помимо этого, размер суперблока отличается от размера блока. Рассчитаем размер одного суперблока.  
%\begin{equation} 
%4  + BlockSize, 
%\end{equation}
%где $BlockSize$ --- размер блока файла. Все выражение измеряется в байтах.
%
%4 байта берутся из одного числа типа $int32$, которое используется для хранения размера XOR блоков. Это число сохраняется в начало файла, чтобы при восстановлении знать, сколько байт считывать из суперблока.

%Итоговая формула вычисления процента избыточной информации выглядит следующим образом:
%\begin{equation}
%\frac{{4 \cdot N + 4 + X}}{{X \cdot 100}},
%\end{equation}
%где $N$ --- количество блоков суперблока, $X$ --- размер исходного файла. 


С целью сократить размер избыточной информации в разработанном методе исследуем, как изменится размер суперблока с применением к нему алгоритмов сжатия.  
\newpage
\section{Исследование изменения размера избыточной информации от применения алгоритмов сжатия}

\begin{flushleft}
Для исследования выберем несколько алгоритмов сжатия:
\end{flushleft}
\begin{itemize}
\item GZIP \cite{gzip};
\item ZSTD \cite{zstd};
\item LZMA \cite{lzma}.
\end{itemize}

%\begin{table}[h]
%\centering
%\caption{Файлы, используемые в исследовании}
%\begin{tabular}{|l|r|}
%\hline
%Имя & Размер (байт) \\
%\hline
%text.txt & 38 \\
%empty\_file.txt & 2000 \\
%cat.jpg & 65955 \\
%war\_and\_peace.txt & 3202332 \\
%ovip.mp4 & 3891031 \\
%video.mp4 & 3891031 \\
%img\_big\_3.jpg & 7054172 \\
%img\_big\_2.jpg & 14733613 \\
%img\_big\_1.jpg & 19290046 \\
%bmstu.png & 303268 \\
%docs.pdf & 10485760 \\
%book2.pdf & 63950001 \\
%filename.txt & 104857600 \\
%book.pdf & 136542178 \\
%video1.mov & 103639311 \\
%\hline
%\end{tabular}
%\end{table}


\begin{flushleft}
В исследовании будем использовать файлы, которые можно разбить на следующие группы:
\end{flushleft}
\begin{itemize}
\item текстовые, с расширением \textit{.txt};
\item документы, с расширением \textit{.pdf};
\item изображения, с расширениями \textit{.jpg} и \textit{.png};
\item видеозаписи, с расширениями \textit{.mov} и \textit{.mp4}.
\end{itemize}

Рассмотрим каждую группу отдельно.

\newpage
\textbf{Текстовые}

\begin{table}[h]
\centering
\caption{Текстовый файл, используемый в исследовании}
\begin{tabular}{|l|r|}
\hline
Имя & Размер (в байтах) \\
\hline
war\_and\_peace.txt & 3 202 332 \\
\hline
\end{tabular}
\end{table}

%\hline
%filename.txt & 5120012 \\
Выполним операцию создания резервной копии. Затем применим алгоримты сжатия и сравним результаты.  

% На гистограмме по оси $X$ указаны названия файлов, по оси $Y$ размер файлов в байтах.

% \begin{flushleft}
% Цветами в легенде отмечены алгоритмы сжатия:
% \end{flushleft}
% \begin{itemize}
% \item синий - ZSTD;
% \item зеленый - GZIP;
% \item желтый - LZMA;
% \item красный - суперблок без сжатия.
% \end{itemize}

%\newpage
%\imgw{empty-file-data}{h!}{1.1\textwidth}{Файл из одинаковых символов}
%
%\textit{filename.txt} --- файл, состоящий из большого количества одинаковых символов. По гистограмме видно, что для таких файлов использование сжатия позволяет экономить место для хранения избыточной информации. Красным отмечен несжатый суперблок, синий еле виден, это размер после сжатия файла. Разница в размерах составляет несколько тысяч процентов.   

% \newpage
Попробуем сжать суперблок книги \textit{«Война и Мир»
}, которая сохранена в файл с расширением \textit{.txt}. 

% \imgw{war-and-peace-file}{h!}{1.1\textwidth}{Книга «Война и мир» в текстовом файле}

Алгоритм сжатия позволяет уменьшить размер суперблока более чем в 2 раза.

Посчитаем, на сколько процентов размер сжатого суперблока меньше размера несжатого.  
Исходный размер измерялся в байтах.  
\begin{table}[h]
\centering
\caption{Сравнение размеров текстовых файлов в разных форматах}
\begin{tabular}{|l|c|r|r|r|}
\hline
Название & Исходный размер & \% с LZMA & \% с GZIP & \% с ZSTD \\
\hline
war\_and\_peace.txt & 3 202 332 & 64.76\% & 62.87\% & 63.67
\% \\
\hline
\end{tabular}
\end{table}
%filename.txt & 5120012 & 576477.93\% & 101892.27\% & 884185.32\% \\


По результатам можно сделать вывод --- при сжатии суперблоков текстовых файлов размер избыточной информации существенно уменьшается: на десятки процентов от исходного.  
\newpage
\textbf{Документы}

\begin{table}[h]
\centering
\caption{Документы, используемые в исследовании}
\begin{tabular}{|l|r|}
\hline
Имя & Размер (в байтах) \\
\hline
book2.pdf & 63 950 013 \\
\hline
book.pdf & 136 542 190 \\
\hline
\end{tabular}
\end{table}

% Гистограмма, построенная для резервных копий документов.

% \imgw{docs-files}{h!}{1.0\textwidth}{Резервные копии документов}

Посчитаем, на сколько процентов размер сжатого суперблока меньше размера несжатого.  
\begin{table}[h]
\centering
\caption{Сравнение размеров документов в разных форматах}
\begin{tabular}{|l|c|r|r|r|}
\hline
Название & Исходный размер & \% с LZMA & \% с GZIP & \% с ZSTD \\
\hline
book2.pdf & 63 950 013 & 76.11\% & 75.93\% & 75.96\% \\
\hline
book.pdf & 136 542 190 & 88.72\% & 88.13\% & 88.46\% \\
\hline
\end{tabular}
\end{table}


По результатам можно сделать вывод --- при сжатии документов размер избыточной информации уменьшается более чем в 2 раза.
   
\textbf{Изображения}

\begin{table}[h]
\centering
\caption{Изображения, используемые в исследовании}
\begin{tabular}{|l|r|}
\hline
Имя & Размер (в байтах) \\
\hline
img5.png & 1 300 370 \\
\hline
img1.jpg & 7 054 172 \\
\hline
img4.png & 8 200 370 \\
\hline
img2.jpg & 14 733 613 \\
\hline
img3.jpg & 19 290 046 \\
\hline
\end{tabular}
\end{table}

% Гистограмма, построенная для резервных копий изображений.

% \imgw{image-files}{h!}{1.0\textwidth}{Резервные копии изображений}

Посчитаем, на сколько процентов размер сжатого суперблока меньше размера несжатого.  
\newpage
\begin{table}[h]
\centering
\caption{Сравнение размеров изображений в формате jpg}
\begin{tabular}{|l|c|r|r|r|}
\hline
Название & Исходный размер & \% с LZMA & \% с GZIP & \% с ZSTD \\
\hline
img1.jpg & 7 054 172 & 0.27\% & 0.20\% & 0.00\% \\
\hline
img2.jpg & 14 733 613 & 1.35\% & 0.10\% & 0.54\% \\
\hline
img3.jpg & 19 290 046 & 0.03\% & 0.01 \% & 0.04 \% \\
\hline
\end{tabular}
\end{table}

По результатам можно сделать вывод --- при сжатии суперблоков изображений в формате jpg размер файла существенно не меняется.

Рассмотрим изборажение в формате png. В сравнении с png, jpg сжимается по алгоритму сжатия с потерями, в то время как для файлов формата png используют алгоритм сжатия без потерь.

Посчитаем, на сколько процентов размер сжатого суперблока меньше размера несжатого.   
\begin{table}[h]
\centering
\caption{Сравнение размеров изображения в формате png}
\begin{tabular}{|l|c|r|r|r|}
\hline
Название & Исходный размер & \% с LZMA & \% с GZIP & \% с ZSTD \\
\hline
img5.png & 1 300 370 & 10.01\% & 9.81\% &  5.02\% \\
\hline
img4.png & 8 200 370 & 32.10\% & 30.37\% &  27.41\% \\
\hline
\end{tabular}
\end{table}

По полученным результатам можно сделать вывод, что сжатие позволяет уменьшить размер суперблоков для изображений, если формат этих файлов использует алгоритм сжатия без потерь. Таким форматом является png. Сравнивая с jpg, разница достигает более 10 \% от размера исходного файла.  
\newpage

\textbf{Видеозаписи}

\begin{table}[h]
\centering
\caption{Видеозаписи, используемые в исследовании}
\begin{tabular}{|l|r|}
\hline
Имя & Размер (в байтах) \\
\hline
video.mp4 & 3 891 031 \\
\hline
video1.mov & 63 639 323  \\
\hline
video2.mp4 & 69 075 086 \\
\hline
\end{tabular}
\end{table}

% Гистограмма, построенная для резервных копий видеозаписей.

% \imgw{video-files-mp4}{h!}{1.1\textwidth}{Резервные копии видеозаписей в формате mp4}

Посчитаем, на сколько процентов размер сжатого суперблока меньше размера несжатого.   

\begin{table}[h]
\centering
\caption{Сравнение размеров видеозаписей в формате mp4}
\begin{tabular}{|l|c|r|r|r|}
\hline
Название & Исходный размер & \% с LZMA & \% с GZIP & \% с ZSTD \\
\hline
video.mp4 & 3 891 031 & 0.29\% & 0.24\% &  0.08\% \\
\hline
video2.mp4 & 69 075 086 & 0.05\% &  0.02\% & 0.04\% \\
\hline
\end{tabular}
\end{table}

По результатам можно сделать вывод --- при сжатии суперблоков видеозаписей формата mp4 размер файла существенно не меняется.

Рассмотрим видеозапись в формате mov. В сравнении с mov, mp4 обычно более сжат и меньше по размеру, в то время как файлы mov часто имеют более высокое качество и больший размер.

% \imgw{video-files-mov}{h!}{1.0\textwidth}{Резервная копия видеозаписи в формате mov}


Посчитаем, на сколько процентов размер сжатого суперблока меньше размера несжатого.   
\begin{table}[h]
\centering
\caption{Сравнение размеров видеозаписи в формате mov}
\begin{tabular}{|l|c|r|r|r|}
\hline
Название & Исходный размер & \% с LZMA & \% с GZIP & \% с ZSTD \\
\hline
video1.mov & 63 639 323 & 26.71\% & 3.25\% &  20.61\% \\
\hline
\end{tabular}
\end{table}

По полученным результатам можно сделать вывод, что сжатие позволяет уменьшить размер суперблоков для видеофайлов, в случае если формат этих файлов имеет минимальный уровень сжатия. Таким форматом является mov. Сравнивая с mp4, разница достигает более 20 \% от размера исходного файла.  

\newpage
\section{Выводы}

\begin{flushleft}
В данном разделе:
\end{flushleft}
\begin{itemize}
\item проведено исследование эффективности разработанного метода --- в результате анализа было получено, что суммарный размер суперблоков равен размеру исходного файла, а метод гарантирует восстановление файла при потере одного сервера, если количество серверов больше одного;
\item проведено исследование изменения размера избыточной информации от применения алгоритмов сжатия;
\item сжатие суперблоков позволяет уменьшить их размер: в худшем случае размер суперблока равен исходному, в лучшем --- уменьшится на десятки процентов;
\item сжатие хорошо работает на файлах, имеющих малый процент сжатия, таких как текстовые файлы, документы, некоторые форматы видео, например, mov;
\item сжатие позволяет уменьшить размер изображений, форматы которых используют алгоритм сжатия без потерь, например, формат png;
\item добавление в разработанный метод алгоритма сжатия позволит уменьшить размер суперблоков и сделает метод более эффективным по памяти. 
\end{itemize}
