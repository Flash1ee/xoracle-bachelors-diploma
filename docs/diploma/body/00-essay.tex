\chapter*{РЕФЕРАТ}
\addcontentsline{toc}{chapter}{РЕФЕРАТ}
Расчетно-пояснительная записка \begin{NoHyper}\pageref{LastPage}\end{NoHyper}~с., \totfig~рис., \tottab~табл., \thetotbibentries~источн., \total{appendixchapters}~прил.

В работе представлена разработка метода резервного копирования с контролируемым размером избыточной информации в распределённом файловом хранилище. 

В аналитическом разделе проведена классификация резервного копирования по способу, а также по месту хранения данных. Описаны основные понятия предметной области резервного копирования, такие как распределенное файловое хранилище, избыточность информации, контрольная сумма и целостность файла. Сформулирован разрабатываемый метод.

В конструкторском разделе приведены схемы алгоритмов, диаграммы последовательности, а также используемые структуры данных.

В технологическом разделе были выбраны средства реализации, а также разработано программное обеспечение, реализующее метод.

В исследовательском разделе проведено исследование эффективности разработанного метода. С целью сокращения размера избыточной информации исследовано влияние алгоритмов сжатия на резервные копии.

КЛЮЧЕВЫЕ СЛОВА

\textit{резервное копирование, распределенное файловое хранилище, избыточность, контрольная сумма, избыточные блоки данных, кодирование с использованием XOR, суперблок, алгоритмы сжатия}