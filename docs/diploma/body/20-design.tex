\chapter{Конструкторский раздел}

\section{IDEF0-диаграмма разрабатываемого метода}

На рисунке ниже представлена диаграмма IDEF0 метода резервного копирования.

\imgw{idef00schema}{h!}{1.0\textwidth}{IDEF0--диаграмма уровня A0}

\begin{flushleft}
На вход метода подаются:
\end{flushleft}
\begin{itemize}
\item место хранения суперблоков --- расположение директорий в ФС;
\item место хранения блоков файлов --- расположение директории в ФС;
\item файлы, для которых делается резервная копия.
\end{itemize}

Выходными данными являются резервные копии файлов (суперблоки).  

Механизмы, используемые методом --- файловая система (ФС) и хранилища данных (директории для хранения).  

\newpage
\section{Схемы алгоритмов работы метода}
Разрабатываемый метод резервного копирования с контролируемым размером избыточной информации в распределенном файловом хранилище имеет следующую схему работы.
\begin{enumerate}
\item Разделение файла на блоки --- исходный файл разбивается на блоки фиксированного размера. Размер зависит от количества серверов, на которые сохраняется суперблок.  

\item Вычисление контрольных сумм --- для каждого блока данных вычисляется контрольная сумма.

\item Создание избыточной информации --- вычисление резервных блоков при помощи операции XOR, применяемой к блокам файла.

\item Распределение избыточной информации --- суперблоки распределяются по разным серверам или узлам в распределенном файловом хранилище.

\item Восстановление данных --- в случае потери или повреждения блока данных, на основе избыточной информации можно восстановить потерянные или поврежденные данные.
\end{enumerate}

Такие слова, как суперблок, избыточная информации и резервная копия в работе являются словами синонимами.  

Рассмотрим детальнее эти шаги.
\newpage
\subsection{Алгоритм создания блоков}
\imgw{blocks-create}{h!}{0.45\textwidth}{Алгоритм создания блоков файлов}

\newpage
\begin{flushleft}
Алгоритм, для одного файла, можно описать следующим образом.
\end{flushleft}

\begin{enumerate}
\item Определить количество блоков --- количество блоков равно количеству используемых серверов, чтобы на каждом сервере хранилась часть файла.

\item Определить размер блока --- для этого используется размер файла и количество блоков. Берется минимальное число $X$, которое больше размера файла, а также кратно количеству блоков и делится на количество блоков.  

\item Чтение файла --- прочитать исходный файл.

\item Вычисление контрольной суммы файла --- для файла вычисляется контрольная сумма, чтобы проверять целостность файла.

\item Создание блоков --- разделить прочитанные данные на блоки заданного размера.

\item Вычисление контрольной суммы --- для каждого созданного блока данных вычисляется контрольная сумма, чтобы проверять целостность каждого блока в отдельности.

\item Сохранение блоков и контрольных сумм --- сохранить каждый блок данных на диск, а также сохранить контрольные суммы в конфигурационный файл.

\end{enumerate}

Метод позволяет подготовить данные к созданию резервной копии. Контрольные суммы обеспечивают возможность проверки целостности файлов и его блоков.

\newpage
\subsection{Алгоритмы создания и восстановления суперблока}

\imgw{superblock}{h!}{0.5\textwidth}{Алгоритм создания суперблока}

\newpage
\begin{flushleft}
Алгоритм можно описать следующим образом.
\end{flushleft}

\begin{enumerate}
\item Выбор блоков данных --- выбрать блоки данных для включения в суперблок. Эти блоки являются блоками файла, для которого делается резервная копия.  

\item Выбрать сервер для сохранения суперблока --- для сохранения суперблока нужно выбрать сервер. Для этого можно использовать алгоритм Round-Robin.  


\item Создание суперблока --- создать суперблок, объединив выбранные блоки данных. С помощью суперблока можно восстановить исходный файл.  

\item Сохранение суперблока и информации о его местоположении --- сохранить созданный суперблок на выбранный сервер, а также информацию о местоположении.  
\end{enumerate}

Выбор сервера осуществляется следующим образом. 

\begin{flushleft}
\textbf{Алгоритм распределения суперблоков по серверам}
\end{flushleft}

\begin{enumerate}
\item Инициализация переменных и структур данных.
\item Получение списка доступных серверов, на которых можно хранить суперблоки. Этот список определен предварительно.  
\item Получение списка суперблоков, которые нужно распределить по серверам.
\item Сохранить суперблок на сервере, используя для выбора сервера алгоритм Round-Robin \cite{round-robin}.  
\end{enumerate}

Рассмотрим алгоритм восстановления суперблока --- операцию, при которой создается файл, позволяющий восстановить исходный файл при его повреждении.  
\newpage
\imgw{blocks-restore}{h!}{0.8\textwidth}{Алгоритм восстановления суперблока}
\newpage

\begin{flushleft}
Алгоритм может быть описан следующим образом.
\end{flushleft}
\begin{enumerate}
\item Получить суперблок восстанавливаемого файла --- проверяются все файлы из конфигурационного файла, если на диске нет блока, который есть в конфиге, или у блока не сходится контрольная сумма с той, которая была получена при его создании, то блок помечается как испорченный.

\item В цикле по всем суперблокам ищем такой, у которого один из блоков является испорченным.

\item Делаем XOR этого суперблока с парным блоком --- блоком, с которым была проведена операция XOR при создании суперблока.  

\item Считаем и сравниваем контрольную сумма восстановленного блока. Если она не равна контрольной сумме, хранимой в конфигурационном файле, то выходим с ошибкой. 

\item Если контрольные суммы сошлись, сохраняем восстановленный блок на место старого блока.  
\end{enumerate}
\textbf{Алгоритм восстановления файла}
%\imgw{file-restore}{h!}{0.5\textwidth}{Алгоритм восстановления файла}

\begin{flushleft}
Алгоритм может быть описан следующим образом.
\end{flushleft}

\begin{enumerate}
\item Из конфигурационного файла извлечь информацию о местоположении блоков файла.

\item Последовательно записать в файл все блоки.

\item Сохранить восстановленный файл.  

\end{enumerate}

%Для последнего блока файла необходимо удалить нули, которые добавляются в суперблок при его создании.  
%
%Дополнение данных нулями используется в контексте методов избыточности, чтобы обеспечить равномерность размера блоков.
%
%Если размер блока данных фиксирован, то для файлов, чей размер не делится нацело на размер блока, может потребоваться дополнение данных нулями до достижения необходимого размера блока. 
%Это позволяет обеспечить однородность блоков данных и упростить последующую обработку блоков.

\newpage

\subsection{Вычисление контрольной суммы}

В методе необходимо предусмотреть проверку целостности файлов и блоков. Она позволит подтвердить, что данные являются верными и не подверглись непредвиденным изменениям.

Для этого используется механизм контрольных сумм. 
Рассмотрим процесс вычисления контрольной суммы для блока.

\imgw{checksum}{h!}{1.0\textwidth}{Вычисление контрольной суммы файла и блоков}

Например, у нас есть файл размером 1 МБ и мы хотим вычислить контрольную сумму для всего файла и его блоков.
\begin{flushleft}
\textbf{Алгоритм вычисления контрольной суммы}
\end{flushleft}
\begin{enumerate}
\item Размер файла --- 1 МБ, размер блока файла --- 512 КБайт.

\item Делим файл на два блока.

\item Для файла и каждого блока вычисляем контрольную сумму, применяя хеш-функцию. Получаем контрольные суммы.

\item Сохраняем контрольные суммы в конфигурационный файл.
\end{enumerate}


Теперь для проверки целостности блока или всего файла достаточно еще раз вычислить контрольную сумму и сравнить с той, что лежит в конфигурационном файле. В конфигурационном файле находятся контрольные суммы, которые были получены при создании резервной копии. Если новая контрольная сумма не совпадает с суммой из конфига, то файл (или блок) нужно восстановить, он поврежден.  

В качестве хеш функции используется алгоритм SHA-256 \cite{sha256}.  

Формула, по которой вычисляется контрольная сумма для блока файла выглядит следующим образом:
\begin{equation}
\label{eq:checksum_block}
h_{i} = \mathrm{sha256}(\mathrm{block}_i), i \in [1,n],
\end{equation}
где $h_{i}$ --- контрольная сумма блока файла, $\mathrm{sha256}$ --- хеш-функция sha256, $\mathrm{block}_i$ --- блок файла, $n$ --- количество блоков файла.  


Формула, по которой вычисляется контрольная сумма для всего файла выглядит следующим образом:

\begin{equation}
\label{eq:checksum_file}
H =  \mathrm{sha256}(\mathrm{file}),
\end{equation}
где $H$ --- итоговая контрольная сумма всего файла, $\mathrm{file}$ --- исходный файл.

\newpage
\section{Структуры данных}
\begin{flushleft}
Основными структурами данных являются:
\end{flushleft}

\begin{itemize}
\item блок данных ---  это базовая структура, представляющая блок информации фиксированного размера, где каждый блок содержит фрагмент данных файла, которые могут быть сохранены на различных серверах.

\item суперблок --- это агрегированная структура, которая содержит информацию о блоках данных.
\end{itemize}

\textbf{Конфигурационный файл}

В контексте метода конфигурационный файл представляет собой место хранения информации о текущем состоянии хранилища: расположение блоков, блоков данных, контрольные суммы, связи между блоками и суперблоками. Всё это позволяет восстановить файлы в случае их повреждения.  
%
%, списка файлов, блоков данных, суперблоков и других параметров, необходимых для восстановления.
%
%Конфигурационный файл хранит информацию о расположении блоков данных, контрольных суммах, связях между блоками и других метаданных, которые позволяют восстановить файлы в случае потери или повреждения данных.

Файл создается и обновляется в процессе работы метода, используется для восстановления данных.

Cтруктура данных конфигурационного файла включает следующие поля:  

\textbf{Servers} -- Список серверов, на которых хранятся блоки файлов.
\begin{table}[h]
    \centering
    \caption{Описание полей поля \texttt{servers}}
    \label{tbl:servers}
        \begin{tabular}{|p{0.2\textwidth}|p{0.6\textwidth}|}
            \hline
            \textbf{Поле} & \textbf{Описание} \\\hline
            \texttt{id} & Идентификатор \\\hline
            \texttt{name} & Название сервера \\\hline
            \texttt{address} & Адрес \\\hline
            \texttt{port} & Порт \\\hline
        \end{tabular}
\end{table}

При использовании абстракции сервер --- это папка, поле servers всегда равно \textbf{null}.  
\newpage
Поля \textbf{files} --- Список файлов, которые были сохранены в хранилище. Рассмотрим структуру данных.

\begin{table}[h]
\centering
\caption{Описание полей списка файлов}
\label{tbl:files}
\begin{tabular}{|p{0.2\textwidth}|p{0.6\textwidth}|}
\hline
\textbf{Поле} & \textbf{Описание} \\\hline
\texttt{filename} & Имя файла \\\hline
\texttt{size} & Размер файла в байтах \\\hline
\texttt{block\_size} & Размер блока в байтах \\\hline
\texttt{blocks} & Список блоков, составляющих файл \\\hline
\texttt{created\_at} & Дата и время создания файла \\\hline
\end{tabular}
\end{table} 

Поле \textbf{blocks} выглядит следующим образом.

\begin{table}[h]
\centering
\caption{Описание полей блоков \textit{blocks}}
\label{tbl:blocks}
\begin{tabular}{|p{0.2\textwidth}|p{0.6\textwidth}|}
\hline
\textbf{Поле} & \textbf{Описание} \\\hline
\texttt{id} & Идентификатор \\\hline
\texttt{checksum} & Контрольная сумма файла \\\hline
\texttt{location} & Местоположение \\\hline
\end{tabular}
\end{table}


\textbf{Superblocks} --- Список суперблоков, которые содержат информацию о блоках, принадлежащих определенным файлам. 


Для списка суперблоков (\texttt{superblocks}) в конфигурационном файле можно использовать следующую таблицу:

\begin{table}[h]
\centering
\caption{Описание полей списка суперблоков}
\label{tbl:superblocks}
\begin{tabular}{|p{0.2\textwidth}|p{0.6\textwidth}|}
\hline
\textbf{Поле} & \textbf{Описание} \\\hline
\texttt{id} & Уникальный идентификатор суперблока \\\hline
\texttt{name} & Имя суперблока \\\hline
\texttt{blocks} & Список блоков, входящих в суперблок \\\hline
\end{tabular}
\end{table}


\textbf{LastUsedServerIndex} -- индекс последнего использованного сервера. Это поле используется для определения следующего сервера, куда будет сохранен суперблок, чтобы обеспечить равномерное распределение суперблоков по серверам и увеличить надежность хранения данных.


Пример конфигурационного файла для метода приведен в листингах \ref{lst:config1} и \ref{lst:config2}.  

%\newpage	
%\listingfile{config.json}{config-1}{Json}{Конфигурационный файл метода}{linerange={1-51}}

%\listingfile{config.json}{config-2}{Json}{Конфигурационный файл метода}{linerange={52-100}}

\newpage
\section{Диаграммы последовательности}

Программно-алгоритмический комплекс, использующий разрабатываемый метод, имеет несколько команд. Основные из них описаны ниже: \textit{add}, \textit{backup}, \textit{restore}.

Рассмотрим их в виде диаграмм последовательности.  

\textbf{Команды разрабатываемого метода}

\imgw{sequence-add}{h!}{1.0\textwidth}{Диаграмма последовательности для команды add}

\newpage
%\subsection{Команда backup}

\imgw{sequence-backup}{h!}{1.0\textwidth}{Диаграмма последовательности для команды backup}

%\subsection{Команда restore}

\imgw{sequence-restore}{h!}{0.9\textwidth}{Диаграмма последовательности для команды restore}
\newpage

\section{Выводы}
\begin{flushleft}
В данном разделе был разработан метод резервного копирования с контролируемым размером избыточной информации в распределенном файловом хранилище, детально рассмотрены алгоритмы:
\end{flushleft}
\begin{itemize}
\item создания блоков;
\item создания и восстановления суперблоков;
\item распределения блоков по разным серверам;
\item восстановления файла.
\end{itemize}

Разработаны структуры данных для хранения информации о файлах, блоках и суперблоках в виде конфигурационного файла.  
