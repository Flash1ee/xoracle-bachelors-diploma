\chapter*{ВВЕДЕНИЕ}
\addcontentsline{toc}{chapter}{ВВЕДЕНИЕ}

В современном мире, где информационные технологии занимают все более важное место, защита данных становится критически важным вопросом. 

Резервное копирование является одним из ключевых инструментов для обеспечения надежности и безопасности данных. Оно позволяет восстановить данные в случае их потери или повреждения.

Существует множество методов резервного копирования, отличающихся друг от друга по способу организации резервных копий, использованию избыточной информации и т.д. 

В рамках работы предстоит разработать метод резервного копирования с контролируемым размером избыточной информации. Этот метод позволяет контролировать размер такой информации, что повышает эффективность использования ресурсов хранения.

Важным элементом резервного копирования является выбор хранилища для резервных копий. Распределенные файловые системы предоставляют множество преимуществ в этом отношении, в том числе возможность распределения данных по нескольким узлам, обеспечения отказоустойчивости и т.д.

