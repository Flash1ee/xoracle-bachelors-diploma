\chapter{Аналитический раздел}
%В данном разделе будет проведен анализ существующих методов резервного копирования, будут рассмотрены основные понятия и термины, а также проведен сравнительный анализ существующих методов. 
%В результате анализа будет сформулирована идея метода резервного копирования с контролируемым размером избыточной информации.

\section{Резервное копирование}

Резервное копирование --– это процесс создания копии данных на устройства хранения, который используется для предотвращения потери данных вследствие их случайного удаления или повреждения носителя информации, например, выхода из строя жесткого диска \cite{acronics-backup-info}.

Основная цель резервного копирования --- обеспечить защиту от потери данных, вызванную сбоями оборудования, программными сбоями, вредоносными программами или человеческими ошибками.
Копирование может быть осуществлено на различных уровнях --- от отдельных файлов и папок до целых систем и баз данных. 
Это необходимо для обеспечения надежности, целостности данных и их доступности в случае непредвиденных обстоятельств.
В процессе резервирования данных создается копия данных, которая хранится отдельно от основных данных. 
Резервные копии могут быть созданы на различных носителях, таких как диски, ленты, хранилища.

Существует несколько классификаций типов резервного копирования. Ниже представлены их примеры.
\begin{enumerate}
	\item По способу хранения и обновления данных \cite{local-cloud-backup}.
	\item По месту хранения данных \cite{backup-types}.
\end{enumerate}

Рассмотрим подробнее данные классификации.

\newpage
\subsection{Классификация по способу хранения и обновления данных}
\subsubsection{Полное копирование}

%Полное копирование (Full backup): при этом типе копирования
При этом типе резервирования копируются все данные и файлы с исходного устройства на резервное устройство. Этот тип копирования является наиболее надежным, но также и наиболее затратным по времени и месту на хранение.

\imgw{backup-full}{h!}{1.0\textwidth}{Процесс полного резервного копирования}

\newpage
\subsubsection{Инкрементное копирование}

%Инкрементное копирование (Incremental backup): при этом типе
При этом типе резервирования копируются только измененные данные с момента последнего копирования. Это позволяет сократить время и место на хранение, но также увеличивает риск потери данных в случае отказа системы.

\imgw{backup-inc}{h!}{1.0\textwidth}{Процесс инкрементного резервного копирования}

\newpage
\subsubsection{Дифференциальное копирование}

%Дифференциальное копирование (Differential backup): при этом типе
При этом типе резервирования копируются только измененные данные с момента последнего полного копирования. Это позволяет сократить время и место на хранение, но также увеличивает риск потери данных в случае отказа системы.

\imgw{backup-diff}{h!}{1.0\textwidth}{Процесс дифференциального резервного копирования}

\newpage

\subsubsection{Сравнение методов}

Сведем разницу между методами резервного копирования по способу хранения и обновления данных в таблицу. 


\begin{table}[h]
  \centering
  \caption{Сравнение полного, инкрементного и дифференциального резервного копирования}
  \begin{tabular}{|l|c|c|c|}
    \hline
    \textbf{Хар-ка} & \textbf{Полное} & \textbf{Инкр-ое} & \textbf{Дифф-ное} \\
    & \textbf{копирование} & \textbf{копирование} & \textbf{копирование} \\
    \hline
    Объем данных & Большой & Постепенно & Постепенно \\
    & & растет & растет \\
    \hline
    Время & Длительное & Быстрое & Среднее \\
    копирования & время & время & \\
    \hline
    Затраты на & Высокие & Низкие (после & Низкие (после \\
    хранилище & & полного & полного \\
    & & копирования) & копирования) \\
    \hline
    Восстановление & Быстрое & Затруднительно & Затруднительно \\
    данных & & без полного & без полного \\
    & & копирования & копирования \\
    \hline
    Доступность & Нет & Да & Да \\
    в управлении & & & \\
    \hline
    Частота & Единоразово & Периодически & Периодически \\
    выполнения & & & \\
    \hline
    Затраты на & Высокие & Низкие & Низкие \\
    сетевой трафик & & & \\
    \hline
  \end{tabular}
\end{table}


Рассмотрим каждую характеристику отдельно.

\begin{flushleft}
\textbf{Объем данных:}  
\end{flushleft}

\begin{itemize}
\item полное копирование имеет большой объем данных, так как копирует все файлы и папки каждый раз;

\item инкрементное и дифференциальное копирование имеют постепенно растущий объем данных, так как они копируют только измененные или добавленные данные, при этом самая первая копия --- полная.
\end{itemize}

\begin{flushleft}
\textbf{Время копирования:}
\end{flushleft}

\begin{itemize}
\item полное копирование требует длительного времени, так как каждый раз выполняется полное сканирование и копирование всех файлов;
\item инкрементное копирование обычно быстрое, так как копируются только новые или измененные файлы;
\item дифференциальное копирование занимает среднее время, так как копируются только измененные файлы с момента последнего полного копирования.
\end{itemize}

\begin{flushleft}
\textbf{Затраты на хранилище:}
\end{flushleft}

\begin{itemize}
\item полное копирование требует высокие затраты на хранилище, так как каждый раз создает полную копию данных;
\item инкрементное и дифференциальное копирование имеют низкие затраты на хранилище, так как они сохраняют только измененные или добавленные данные. 
\end{itemize}

\begin{flushleft}
\textbf{Восстановление данных:}
\end{flushleft}

\begin{itemize}
\item полное копирование обеспечивает быстрое восстановление данных, так как все данные находятся в одном месте;
\item инкрементное и дифференциальное копирование могут быть затруднительными при восстановлении данных без полного копирования, так как они зависят от последовательности и наличия предыдущих копий.
\end{itemize}

\begin{flushleft}
\textbf{Доступность в управлении:}
\end{flushleft}

\begin{itemize}
\item полное копирование обычно менее сложное в управлении, так как требует полного сканирования и копирования всех файлов каждый раз;
\item инкрементное и дифференциальное копирование более простые в управлении, так как копируют только измененные или добавленные данные.
\end{itemize}

\begin{flushleft}
\textbf{Затраты на сетевой трафик:}
\end{flushleft}

\begin{itemize}
\item полное копирование требует высокие затраты на сетевой трафик, так как каждый раз передает все данные;
\item инкрементное и дифференциальное копирование имеют более низкие затраты на сетевой трафик, так как передают только измененные или добавленные данные;
\item однако, в случае инкрементного и дифференциального копирования, для полного восстановления данных может потребоваться передача большого количества инкрементных или дифференциальных копий, что может увеличить общий сетевой трафик.
\end{itemize}

\begin{flushleft}
\textbf{Гибкость восстановления:}
\end{flushleft}

\begin{itemize}
\item полное копирование обеспечивает полную гибкость восстановления данных, так как все данные находятся в одной копии;
\item инкрементное копирование обеспечивает гибкость восстановления данных только до момента последнего полного копирования, требуя последовательного восстановления каждой инкрементной копии;
\item дифференциальное копирование обеспечивает гибкость восстановления данных до момента последнего полного или дифференциального копирования, требуя последовательного восстановления каждой дифференциальной копии.
\end{itemize}

Для метода используем механизм полного копирования, так как он более защищен от ошибок и повреждений --- другие типы должны иметь доступ к нескольким последовательным копиям, 
что увеличивает риск появления ошибок при восстановлении. 

\newpage

\subsection{Классификация по месту хранения данных}

\subsubsection{Локальное копирование}

Локальное резервное копирование представляет собой тип резервного копирования, при котором данные копируются с одного устройства хранения на другое на локальном уровне. Это может быть копирование данных с жесткого диска на внешний носитель, на другой компьютер или на локальный сервер.

При локальном копировании данные остаются в пределах локальной сети и не передаются через общедоступную (интернет).

\subsubsection{Копирование в удаленное хранилище}

Копирование в удаленное хранилище представляет собой тип резервного копирования, при котором данные копируются с одного устройства в удаленное хранилище с использованием сети. В этом случае данные копируются на удаленный сервер или хранилище, находящееся в другом физическом или логическом месте.

\subsubsection{Различия сетей}

Локальная сеть и удаленная сеть --- это два разных понятия, связанных с организацией и обеспечением сетевого соединения.

Локальная сеть (от англ. Local Area Network) представляет собой сеть, ограниченную географически, которая охватывает небольшую географическую область, такую как офис, университет или здание \cite{lan-cloudflare}.

Локальная сеть строится для обеспечения связи и обмена данными между устройствами, находящимися в пределах этой области. Обычно она использует проводные или беспроводные технологии связи, такие как Ethernet или Wi-Fi. Локальная сеть может быть настроена для обмена файлами, печати, доступа к общим ресурсам и других совместных задач.

С другой стороны, глобальная сеть (от англ. Wide Area Network) --- это сеть, которая объединяет компьютеры и устройства на больших расстояниях, как правило, через Интернет. 

Удаленная сеть обеспечивает связь между географически разделенными локальными сетями. Она может использовать различные технологии, например, виртуальные частные сети (VPN). Удаленная сеть позволяет расширить границы локальных сетей и обеспечить доступ к общим ресурсам и услугам, несмотря на физическое расстояние.

Таким образом, \textit{разница между локальной и удаленной сетью} заключается в их географической области охвата и способе организации соединения. Локальная сеть предназначена для связи устройств внутри ограниченной области, в то время как удаленная сеть предоставляет связь между удаленными локациями или сетями через общедоступные сети. Однако в реальных сетевых сценариях часто используется комбинация локальных и удаленных сетей, чтобы обеспечить эффективную связь и доступность между удаленными и локальными устройствами.

\begin{table}[h]
  \centering
  \caption{Сравнение локального и удаленного резервного копирования}
  \begin{tabular}{|l|c|c|}
    \hline
    \textbf{Характеристика} & \textbf{Локальное} & \textbf{Удаленное} \\
    & \textbf{копирование} & \textbf{копирование} \\
    \hline
    Скорость доступа & Высокая & Ограничена \\
     & & интернет-соединением \\
    \hline
    Надежность & Зависит от & Высокий уровень \\
     & носителей & надежности \\
    \hline
    Масштабируемость & Ограничена & Поддерживает \\
     & объемом хранилища & большие объемы \\
    \hline
    Управление данными & Полный контроль & Доверенность \\
     & и доступ & и управление \\
    \hline
    Восстановление & Быстрое с локальных & Восстановление \\
    данных & носителей & через интернет \\
    \hline
    Стоимость & Высокие затраты & Ежемесячная \\
     & на оборудование & плата \\
  	\hline
  \end{tabular}
\end{table}

Рассмотрим каждую характеристику отдельно.
\begin{flushleft}
\textbf{Скорость доступа:}
\end{flushleft}

\begin{itemize}
\item локальное копирование обеспечивает высокую скорость доступа к данным, поскольку они хранятся на локальных носителях, доступ к которым осуществляется через локальную сеть;
\item удаленное копирование ограничено скоростью интернет-соединения, что может влиять на время доступа к данным, особенно при крупных объемах данных или медленном интернет-соединении.
\end{itemize}

\begin{flushleft}
\textbf{Надежность:}
\end{flushleft}

\begin{itemize}
\item локальное копирование зависит от надежности физических носителей, на которых хранятся резервные копии и в случае повреждения или отказа носителей, данные могут быть потеряны;
\item удаленное копирование предоставляет высокий уровень надежности, так как облачные провайдеры обеспечивают резервное копирование и репликацию данных на удаленных серверах, что обеспечивает сохранность данных даже в случае сбоя оборудования или физической гибели данных.
\end{itemize}

\begin{flushleft}
\textbf{Масштабируемость:}
\end{flushleft}

\begin{itemize}
\item локальное копирование ограничено объемом доступного физического хранилища, для увеличения масштабируемости необходимо приобретать и устанавливать дополнительное оборудование;
\item удаленное копирование позволяет масштабировать хранилище данных в зависимости от потребностей, так как поставщики услуг предлагают гибкую масштабируемость, позволяющую увеличивать или уменьшать объем хранилища по требованию.
\end{itemize}

\begin{flushleft}
\textbf{Стоимость:}
\end{flushleft}

\begin{itemize}
\item локальное копирование включает высокие затраты на приобретение и обслуживание оборудования для хранения и обработки данных;
\item удаленное копирование предлагает модель ежемесячной платы за использование облачного хранилища, что может быть более экономически выгодным, особенно для небольших и средних предприятий.
\end{itemize}


Для метода объединим локальное копирование и удаленное, то есть сделаем возможность использования обоих способ копирования. Таким образом удастся взять преимущества обоих методов.


Резервное копирование является неотъемлемой частью обеспечения безопасности данных в любой информационной системе. 
Однако, с ростом объемов данных и требований к их доступности, возникает необходимость в использовании распределенных файловых хранилищ.
Такие хранилища позволяют хранить большие объемы данных на нескольких устройствах, обеспечивая высокую доступность и отказоустойчивость системы. 
При этом, для обеспечения безопасности данных, необходимо применять методы резервного копирования.

\section{Распределенное файловое хранилище}

Распределенное файловое хранилище (РФХ) --- это способ организации хранения данных, при котором файлы разбиваются на части и хранятся на различных узлах (серверах) в сети \cite{distributed-fs}. 

В рамках работы используется распределенное файловое хранилище, которое работает не с файлами, а с его блоками. 

\subsection{Блочное файловое хранилище}
При запросе на доступ к файлу, клиентское приложение обращается к одному из узлов, который предоставляет запрошенную часть файла (блок).  
В РФХ обычно применяются технологии, позволяющие обеспечить надежность и доступность данных, такие как репликация \cite{replication}, шардинг \cite{sharding}, избыточность информации и т.д.

Кроме того, в распределенных хранилищах широко используется понятие избыточности информации. 

Избыточность информации обеспечивает сохранность данных даже в случае сбоев в системе или потери части информации. 
Она достигается путем создания дополнительных копий данных или использованием специальных алгоритмов, таких как кодирование с избыточностью.
\newpage

На рисунке ниже показана высокоуровненвая архитектура такого хранилища.  

\imgw{dfs-flow}{h!}{1.0\textwidth}{Схема работы распределенного файлового хранилища}

Клиент, посредством сети, отправляет файл в хранилище. Файл разбивается на блоки и распределяется сервером между различными хранилищами данных. Каждое такое хранилище может содержать несколько копий блоков данных или их фрагментов. 

Это обеспечивает избыточность данных, так как в случае отказа одного хранилища, данные остаются доступными через другие копии, размещенные на других узлах.

При запросе файла клиентом, сервер собирает файл из блоков, на которые был разделен и отдает их клиенту. 

Существует большое количество распределенных файловых хранилищ, например, HDFS \cite{hdfs}, Ceph \cite{ceph}.  

\subsection{Избыточность информации}
Избыточность информации --- это концепция хранения дополнительных дублирующих данных в целях повышения надежности системы. 
В информационных системах, избыточность может быть реализована через различные методы, например, резервирование данных.

Избыточность информации играет важную роль в распределенных системах, где различные компоненты могут находиться на разных узлах сети и могут быть недоступны в случае сбоев в сети или отказов оборудования. 

При использовании избыточности данных в распределенных системах, дополнительные копии данных хранятся на разных узлах, что повышает доступность данных и уменьшает риск потери данных.

Однако использование избыточности данных может повлечь за собой дополнительные расходы на хранение и передачу данных, а также на управление и синхронизацию дополнительных копий. 
Поэтому необходим баланс между уровнем избыточности и экономическими затратами на его реализацию и поддержание.

Рассмотрим механизмы, которые создают избыточность в РФХ.

\subsubsection{Репликация данных}
Этот механизм заключается в создании нескольких копий одних и тех же данных и их распределении по разным серверам в сети. Если один сервер становится недоступным или данные на нем повреждены, то данные могут быть восстановлены из других копий. Репликация обеспечивает высокую доступность данных и устойчивость к сбоям.

\subsubsection{Разделение информации}
Этот механизм предполагает разбиение данных на части и их распределение по нескольким серверам. При чтении или записи данных они обрабатываются параллельно на нескольких серверах, что увеличивает производительность. Если один сервер становится недоступным, данные всё еще доступны на других серверах. Полосное размещение также позволяет увеличить пропускную способность системы.
\newpage
\subsubsection{Избыточные блоки данных}
Этот механизм заключается в добавлении дополнительных избыточных блоков данных, которые могут использоваться для восстановления потерянных или поврежденных данных. 

Избыточные блоки (суперблоки) могут быть созданы путем применения операций кодирования, таких как XOR или RAID \cite{raid}. 
Эти блоки содержат дополнительную информацию, которая позволяет восстановить данные, если один или несколько блоков становятся недоступными.
\newpage

\begin{table}
\centering
\caption{Сравнение механизмов избыточности в распределенном файловом хранилище}
\begin{tabular}{|l|c|c|} 
\hline
\textbf{Механизм}                                                            & \textbf{Плюсы}                                                          & \textbf{Минусы}                                                                 \\ 
\hline
Репликация                                                                   & \begin{tabular}[c]{@{}c@{}}Увеличение доступности \\данных\end{tabular} & \begin{tabular}[c]{@{}c@{}}Высокие затраты на \\оборудование\end{tabular}       \\
                                                                             & Быстрый доступ к данным                                                 &                                                                                 \\ 
\hline
\begin{tabular}[c]{@{}l@{}}Разделение \\информации\end{tabular}              & Экономия места                                                          & \begin{tabular}[c]{@{}c@{}}Зависимость от точности \\фрагментации\end{tabular}  \\
                                                                             & Высокая надежность                                                      & Увеличение нагрузки на сеть                                                     \\ 
\hline
\begin{tabular}[c]{@{}l@{}}Избыточные \\блоки \\данных\end{tabular} & Экономия места                                                          & Вычислительная сложность                                                        \\
                                                                             & Высокая надежность                                                      & \begin{tabular}[c]{@{}c@{}}Увеличение нагрузки на \\процессор\end{tabular}      \\
\hline
\end{tabular}
\end{table}

Рассмотрим плюсы и минусы отдельно.
\begin{flushleft}
\textbf{Репликация}

Плюсы:
\end{flushleft}

\begin{itemize}
\item увеличение доступности данных --- наличие нескольких копий данных обеспечивает доступ к информации, даже если одна из копий недоступна;

\item быстрый доступ к данным --- каждая реплика содержит полные данные, что позволяет быстро получать данные без дополнительных операций.
\end{itemize}

\begin{flushleft}
Минусы:
\end{flushleft}

\begin{itemize}
\item высокие затраты на оборудование --- требуется выделение дополнительного пространства и ресурсов для хранения и поддержки нескольких копий данных.
\end{itemize}

\begin{flushleft}
\textbf{Разделение информации}

Плюсы:
\end{flushleft}

\begin{itemize}
\item экономия места --- разделение информации на фрагменты позволяет сохранять данные в более компактном формате и эффективно использовать доступное пространство;
\item высокая надежность --- при потере или повреждении одного фрагмента данных, остальные фрагменты могут быть использованы для восстановления полной информации.
\end{itemize}

\begin{flushleft}
Минусы:
\end{flushleft}

\begin{itemize}
\item зависимость от точности фрагментации --- некорректное разделение информации на фрагменты может привести к потере данных или затруднить их восстановление;
\item увеличение нагрузки на сеть --- при передаче и обработке фрагментов данных возникает дополнительная нагрузка на сеть, что может снизить производительность.
\end{itemize}

\begin{flushleft}
\textbf{Избыточные блоки данных}

Плюсы:
\end{flushleft}
\begin{itemize}
\item экономия места --- кодирование данных с помощью XOR позволяет сократить объем хранимых данных, используя математические операции для создания избыточной информации;
\item высокая надежность --- при потере или повреждении одного или нескольких фрагментов данных, можно использовать оставшиеся фрагменты и XOR-операции для восстановления полной информации.
\end{itemize}

\begin{flushleft}
Минусы:
\end{flushleft}

\begin{itemize}
\item вычислительная сложность --- выполнение XOR-операций для кодирования и восстановления данных требует вычислительных ресурсов;
\item увеличение нагрузки на процессор --- при выполнении операций кодирования и декодирования с использованием XOR возникает дополнительная нагрузка на процессор, что может повлиять на производительность системы.
\end{itemize}

Анализируя плюсы и минусы, для метода предпочтительнее использовать механизм избыточных блоков, так как он позволяет сэкономить объем занимаемой памяти, а также обеспечить высокую надежность.  

\textbf{Избыточность данных с помощью XOR}

Основная идея метода заключается в применении операции XOR (исключающее ИЛИ) для создания резервных копий блоков файлов. При этом для каждого блока файлов выбирается некоторое количество других блоков, с которыми производится операция XOR. Результат операции сохраняется в отдельный блок, который может использоваться для восстановления данных в случае потери или повреждения исходных блоков.

\textbf{Преимущества использования XOR для создания резервных копий:}
\begin{itemize}
\item XOR позволяет сократить объем хранимых данных, так как резервные блоки не являются полными копиями исходных блоков, а представляют собой только изменения в данных;
\item при использовании XOR можно восстановить исходные данные путем выполнения операции XOR над доступными блоками и резервными блоками.
\end{itemize}

\textbf{Недостатки использования XOR для создания резервных копий:
}
\begin{itemize}
\item если несколько блоков данных повреждены или потеряны, восстановление может быть невозможным.
\end{itemize}

Для обеспечения целостности данных при использовании XOR для создания резервных копий также часто используются контрольные суммы.  

\subsection{Проверка целостности файлов}

Для проверки целостности файлов можно использовать механизмы контрольных сумм или хеширования.

Контрольная сумма --- это значение, рассчитанное по набору данных путём применения определённого алгоритма и используемое для проверки целостности данных при их передаче или хранении. 

Хеширование --- это процесс преобразования входных данных произвольной длины в фиксированную строку фиксированной длины, называемую хешем. Хеш-функции обладают свойством, что даже незначительное изменение входных данных приведет к существенному изменению выходного хеша. Хеширование широко используется для хранения паролей, проверки целостности данных, поиска дубликатов.

Сравнение контрольной суммы (хеша), полученной из исходной версии файла, с контрольной суммой (хешом), находящейся в резервной копии  файла, помогает убедиться, что исходная копия файла является подлинной и не содержит ошибок.

Для вычисления контрольной суммы используют математические алгоритмы, которые отображают данные произвольного размера в массив фиксированного.  

Существуют различные способы вычисления контрольных сумм и хешей. Ниже представлены их примеры.
\begin{enumerate}
\item CRC (Циклический избыточный код) --- CRC-алгоритмы основаны на делении полиномов с двоичными коэффициентами. CRC вычисляет контрольную сумму путем деления данных на заданный полином. Полученный остаток является контрольной суммой \cite{crc}.
\item MD5 --- является одним из алгоритмов хеширования, который принимает произвольный объем данных и вычисляет 128-битную контрольную сумму \cite{md5}.
\item SHA --- семейство алгоритмов SHA, которое используется для вычисления контрольных сумм, например, SHA-256 \cite{sha256}.
\end{enumerate}

\newpage
\section{Выводы}

Определим метод резервного копирования, который будет реализован в рамках дипломного проекта. 

Данный метод основан на использовании распределенного файлового хранилища и контролируемом размере избыточной информации. Он объединяет преимущества различных методов, рассмотренных ранее.  

По способу хранения и обновления данных будет использоваться механизм полного копирования, по месту хранения данных - облачное копирование. Контролируемый размер избыточной информации будет обеспечиваться с помощью  механизма создания избыточных блоков данных (кодирование с использованием XOR).  

Цель работы --- разработать метод резервного копирования с контролируемым размером избыточной информации в распределённом файловом хранилище.

\begin{flushleft}
Для достижения поставленной цели потребуется:
\end{flushleft}
\begin{itemize}
\item описать основные понятия предметной области резервного копирования;
\item провести анализ существующих методов;
\item разработать метод резервного копирования с контролируемым размером избыточной информации в распределенном файловом хранилище;
\item разработать программный комплекс, реализующий интерфейс для взаимодействия с разработанным методом;
\item провести исследование эффективности разработанного метода и выявить количество избыточной информации, хранящейся при резервировании файлов.
\end{itemize}