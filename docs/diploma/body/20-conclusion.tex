\chapter*{ЗАКЛЮЧЕНИЕ}
\addcontentsline{toc}{chapter}{ЗАКЛЮЧЕНИЕ} 

\begin{flushleft}
В ходе выполнения работы были выполнены следующие задачи:
\end{flushleft}
\begin{itemize}
\item описаны основные понятия предметной области резервного копирования;
\item разработан метод резервного копирования с контролируемым размером избыточной информации в распределенном файловом хранилище;
\item разработан программный комплекс, реализующий интерфейс для взаимодействия с разработанным методом;
\item проведено исследование эффективности разработанного метода;
\item анализ показал, что разработанный метод гарантирует восстановление файла при потере одного сервера, если количество серверов больше одного, а также гарантирует восстановление файла при потере одного сервера.   
\end{itemize}

Исследование показало, что добавление в разработанный метод алгоритма сжатия позволит уменьшить размер суперблоков и сделает метод более эффективным по памяти.  

Таким образом, поставленная цель работы, разработать метод резервного копирования с контролируемым размером избыточной информации в распределенном файловом хранилище, была достигнута.
